\documentclass[a4paper,12pt,titlepage,finall]{article}

\usepackage{cmap}
\usepackage{color}
\usepackage[utf8x]{inputenc}
\usepackage[english,russian]{babel}
\usepackage[T2A]{fontenc}
\newcommand{\gt}{\textgreater} % знак больше
\newcommand{\lt}{\textless}       % знак меньше]
\usepackage{geometry}		 % для настройки размера полей
\usepackage{indentfirst}         % для отступа в первом абзаце секции
\usepackage{amsmath}
\usepackage{fancyvrb}
\usepackage{xcolor}
\usepackage{hyperref}
\usepackage{graphicx}
\graphicspath{{/home/artem/Pictures/}}


\usepackage{listings}

\lstset{
inputencoding=utf8x,
extendedchars=false,
keepspaces = true,
language=C++,
basicstyle=\ttfamily,
keywordstyle=\color[rgb]{0,0,1},
commentstyle=\color[rgb]{0.026,0.112,0.095},
stringstyle=\color[rgb]{0.627,0.126,0.941},
numberstyle=\color[rgb]{0.205, 0.142, 0.73},
morecomment=[l][\color{magenta}]{\#},
frame=shadowbox,
escapechar=`,
numbers=left,
breaklines=true,
basicstyle=\ttfamily,
literate={\ \ }{{\ }}1,
tabsize=2,
basicstyle=\footnotesize,
}


% выбираем размер листа А4, все поля ставим по 3см
\geometry{a4paper,left=10mm,top=10mm,bottom=10mm,right=10mm}

\setcounter{secnumdepth}{0}      % отключаем нумерацию секций


\begin{document}
% Титульный лист
\begin{titlepage}
    \begin{center}
	{\small \sc Московский государственный университет \\имени М.~В.~Ломоносова\\
	Факультет вычислительной математики и кибернетики\\}
	\vfill
	{\Large \sc Компьютерный практикум по учебному курсу ""}\\
	~\\
	{\large \bf <<Введение в численные методы \\
	Задание 1>>}\\
	~\\
	{\large \bf  ОТЧЕТ \\ }
	~\\
	{\small \bf  о выполненном задании \\ }
	~\\
	{\small \sc студента 206 учебной группы факультета ВМК МГУ\\}

	{\small \sc Николайчука Артёма Константиновича\\}
	\vfill
    \end{center}

    \begin{center}
	\vfill
	{\small гор. Москва\\2020 г.}
    \end{center}
\end{titlepage}

% Автоматически генерируем оглавление на отдельной странице
\tableofcontents
\newpage

\section{Цели}

Целью данной работы является реализация численных методов нахождения 
решения заданных систем линейных алгебраических уравнений методом Гаусса, в том числе методом Гаусса с выбором главного элемента, и методом верхней релаксации. 

\begin{itemize}
\item Решить заданную СЛАУ  методом Гаусса и методом Гаусса с выбором главного элемента
\item Вычислить определителю матрицы $det(A)$
\item Вычислить обратную матрицу  $A ^ {-1}$
\item Исследовать вопрос вычислительной  устойчивости метода Гаусса
\item Решить заданную СЛАУ итерационным методом верхней релаксации
\item Разработать критерий остановки итерационного процесса для гарантированного получения приближенного решения исходной СЛАУ с заданной точностью
\item Изучить скорость сходимости итераций к точному решению задачи при различных итреационных параметрах $\omega$
\item Проверить правильность решения СЛАУ на различных тестах, используя wolframalpha.com
\end{itemize}

\newpage
-
\section{Постановка задачи}

Дана система линейных уравнений $A\overline{x}=\overline{f}$  порядка $n * n$ с невырожденной матрицей $A$. \\
Написать программу, решающую СЛАУ заданного пользователем размера методом Гаусса и методом Гаусса с выбором главного элемента.\\
Написать программу численного решения СЛАУ заданного пользователем размера, использующую
итерационный метод верхней релаксации. Итерационный процесс имеет вид:
$$ (D + \omega A ^ {(-)}) \frac{x^{k+1}-x^{k}}{\omega} + Ax^{k} = f),$$
где $\omega$ -итерационный параметр.\\
Предусмотреть возможность задания элементов матрицы системы и ее правой части как во входном файле, так и с помощью специальных формул.

\newpage


\section{Описание алгоритмов}

\begin{itemize}
\item Метод Гаусса\\
Цель - найти решение системы.\\
Решения системы линейных алгебраических уравнений методом Гаусса
производится в два этапа:"прямой" и "обратный" ход.
В результате "прямого хода" матрица коэффициентов, входящая в расширенную матрицу,
путем линейных пребразований строк и их перестановок 
последовательно приводится к верхнему треугольному виду.
На этапе "обратного хода" выполняется восстановление решения путем
прохода по строкам матрицы в обратном направлении.\\
В методе Гаусса с выбором главного элемента
на каждой итерации выбирается максимальный из всех элементов подматрицы, что позволяет
уменьшить погрешность вычислений.
 
После приведения матрицы к диагональному виду с помощью классического метода Гаусса, определитель вычисляется, как произведение элементов, стоящих на диагонали.

Задача нахождения обратной матрицы решается с помощью метода Гаусса-Жордана. 
Над расширенной матрицей, составленной из столбцов исходной матрицы и единичной
того же порядка, производятся преобразования метода Гаусса, в результате которых
исходная матрица принимает вид единичной матрицы, а на месте единичной образуется
матрица, обратная исходной.\\
\item Метод верхней релаксации

Метод вехний релакцации это стационарным итерационным методом, 
в котором каждый следующий вектор приближения точного решения вычисляется по формуле:\\
$ (D + \omega T) \frac{y_{k+1} - y_k}{\omega} + A y_k = f $

где 

\begin{description}
    \item[$\omega$] - итерационный параметр, \\
    \item[$y_k$] - $k$-й вектор приближения, \\
    \item[$y_{k+1}$] - ($k+1$)-й вектор приближения, \\
    \item[$A$] - матрица коэффициентов СЛАУ, \\
    \item[$f$] - правая часть СЛАУ, \\
    \item[$T$] - нижняя диагональная матрица матрицы $A$, \\
    \item[$D$] - матрица диагональных элементов матрицы $A$. \\
\end{description}
Метод применим только к положительноопределённым матрицам.
По теореме Самарского для положительно определенных матриц $A$
метод верхней релаксации сходится, если $0 < \omega < 2$. Значением по умолчанию является значение $\omega = \frac{4}{3}$.
За вектор начального приближения берется нулевой вектор.
\end{itemize}
\newpage

\section{Описание и код программы}
Весь код написан на языке программирования python.\\ 
Программа размещена на гитхабе. \href{https://github.com/Strannik2001/Computing_Methods}{https://github.com/Strannik2001/Computing\_Methods}\\

Код основной программы:
\begin{lstlisting}
import sys
import random
from math import sin


class Test:
    def __init__(self, mode, n, a, b=1):
        self.mode = mode
        self.n = n
        self.accuracy = 1e-9
        self.max_iterations = 1000
        self.omega = 4 / 3
        self.need_optimization = b
        self.matrix = []
        self.b = []
        self.q = 0
        self.determinant = 1
        self.invert_matrix = []
        for i in range(n):
            self.invert_matrix.append([])
            self.invert_matrix[i] = [0] * self.n
            self.invert_matrix[i][i] = 1
        self.prepare_matrix(a)
        self.matrix_copy = []
        for i in range(n):
            self.matrix_copy.append([])
            for j in range(n):
                self.matrix_copy[i].append(self.matrix[i][j])

    def prepare_matrix(self, a):
        if self.mode == 'static':
            with open(a, "r") as f:
                for i in range(self.n):
                    self.matrix.append(list(map(float, f.readline().split())))
                self.b = list(map(float, f.readline().split()))
        else:
            for i in range(self.n):
                self.matrix.append([])
                self.matrix[i] = [0] * self.n
            self.b = [0] * self.n
            self.q = 1.001 - 2 * a / 1000
            x = random.random()
            for i in range(self.n):
                self.b[i] = abs(x - self.n / 10) * (i + 1) * sin(x)
            for i in range(1, self.n + 1):
                for j in range(1, self.n + 1):
                    if i == j:
                        self.matrix[i - 1][j - 1] = pow(self.q - 1, i + j)
                    else:
                        self.matrix[i - 1][j - 1] = pow(self.q, i + j) + 0.1 * (j - i)

    def out_matrix(self, a='matrix'):
        if a == 'invert matrix':
            print("Invert matrix")
            m = self.invert_matrix
        else:
            print("Matrix")
            m = self.matrix
        out_matrix(m)

    def get_invert_matrix(self):
        if self.determinant != 0:
            return self.invert_matrix
        else:
            print("No such matrix")
            return None

    def gaus(self):
        sign = 1
        matrix_copy = []
        b_copy = []
        for i in range(self.n):
            b_copy.append(self.b[i])
            matrix_copy.append([])
            for j in self.matrix[i]:
                matrix_copy[i].append(j)
        for i in range(self.n):
            if self.need_optimization:
                maxi = i
                for j in range(i, self.n):
                    if abs(self.matrix[maxi][i]) < abs(self.matrix[j][i]):
                        maxi = j
                if maxi != i:
                    sign *= -1
                self.b[i], self.b[maxi] = self.b[maxi], self.b[i]
                self.matrix[i], self.matrix[maxi] = self.matrix[maxi], self.matrix[i]
                self.invert_matrix[i], self.invert_matrix[maxi] = self.invert_matrix[maxi], self.invert_matrix[i]
            cur = self.matrix[i][i]
            if cur == 0:
                self.determinant = 0
                continue
            self.determinant *= cur
            self.matrix[i] = [z / cur for z in self.matrix[i]]
            self.b[i] /= cur
            self.invert_matrix[i] = [z / cur for z in self.invert_matrix[i]]
            for j in range(i + 1, self.n):
                cur = self.matrix[j][i]
                self.b[j] -= cur * self.b[i]
                self.matrix[j] = [self.matrix[j][z] - cur * self.matrix[i][z] for z in range(self.n)]
                self.invert_matrix[j] = [self.invert_matrix[j][z] - cur * self.invert_matrix[i][z] for z in range(self.n)]
        self.determinant *= sign
        print("Determinant = {}".format(self.determinant))
        for i in range(self.n)[::-1]:
            for j in range(0, i):
                self.invert_matrix[j] = [self.invert_matrix[j][z] - self.matrix[j][i] * self.invert_matrix[i][z] for z in range(self.n)]
                self.b[j] -= self.b[i] * self.matrix[j][i]
                self.matrix[j] = [self.matrix[j][z] - self.matrix[j][i] * self.matrix[i][z] for z in range(self.n)]
        self.matrix = matrix_copy
        if self.determinant == 0:
            self.b = b_copy
            print("No invert_matrix(((")
            return
        print("^^^^^^^^^^^^^^^^^^^^^^^^^^^^^^^^^^")
        print("Answer by gaus method")
        for j in range(self.n - 1):
            print("%015.10f" % self.b[j], end='  |^|  ')
        print("%015.10f" % self.b[self.n - 1])
        print("^^^^^^^^^^^^^^^^^^^^^^^^^^^^^^^^^^")
        self.b = b_copy

    def relax_method(self):
        res = [0] * self.n
        t = [0] * self.n
        for k in range(self.max_iterations):
            for i in range(self.n):
                s1 = 0
                s2 = 0
                for j in range(i):
                    s1 += self.matrix[i][j] * t[j]
                for j in range(i, self.n):
                    s2 += self.matrix[i][j] * res[j]
                t[i] = res[i] + (self.omega / self.matrix[i][i]) * (self.b[i] - s1 - s2)
            d = 0
            for i in range(self.n):
                d += (res[i] - t[i]) * (res[i] - t[i])
            if d < self.accuracy:
                print("^^^^^^^^^^^^^^^^^^^^^^^^^^^^^^^^^^")
                print("Answer by relax method")
                for j in range(self.n - 1):
                    print("%015.10f" % t[j], end='  |^|  ')
                print("%015.10f" % t[self.n - 1])
                print("^^^^^^^^^^^^^^^^^^^^^^^^^^^^^^^^^^")
                return t
            for i in range(self.n):
                res[i] = t[i]
        print("^^^^^^^^^^^^^^^^^^^^^^^^^^^^^^^^^^")
        print("Answer by relax method")
        for j in range(self.n - 1):
            print("%015.10f" % res[j], end='  |^|  ')
        print("%015.10f" % res[self.n - 1])
        print("^^^^^^^^^^^^^^^^^^^^^^^^^^^^^^^^^^")
        return res


def out_matrix(a):
    if a is None:
        return
    n = len(a)
    if n > 20:
        print('too much size')
        return
    print("##########################################")
    for i in range(n):
        for j in range(n - 1):
            print("%015.10f" % a[i][j], end='  |^|  ')
        print("%015.10f" % a[i][n - 1])
    print("##########################################")


def mul_matrix(a, b):
    n = len(a)
    res = []
    for i in range(n):
        res.append([])
        for j in range(n):
            res[i].append(0)
            for k in range(n):
                res[i][j] += a[i][k] * b[k][j]
    return res


def out_usage():
    print("##Usage##")
    print("First argument - 'static' or 'dynamic'")
    print("Second argument - matrix order")
    print("Third argument - path_to_file if static, starting number if dynamic")


def main():
    try:
        if sys.argv[1] == 'static':
            matrix = Test('static', int(sys.argv[2]), sys.argv[3])
        elif sys.argv[1] == 'dynamic':
            matrix = Test('dynamic', int(sys.argv[2]), float(sys.argv[3]))
        else:
            out_usage()
            exit(0)
    except:
        out_usage()
        print("Type parameters")
        a = input().split()
        if a[0] == 'static':
            matrix = Test('static', int(a[1]), a[2])
        elif a[0] == 'dynamic':
            matrix = Test('dynamic', int(a[1]), float(a[2]))
        else:
            out_usage()
            exit(0)
    matrix.gaus()
    matrix.relax_method()
    out_matrix(matrix.get_invert_matrix())


if __name__ == "__main__":
    main()

\end{lstlisting}
\newpage

\section{Тестирование}

Тестирование проводится на наборах СЛАУ из приложения 1-11 и приложения 2-5 и тестах, которые я придумал сам. \\
Результат работы на невырожденных матрицах сравниваем с точным ответом, полученном на сайте wolframalpha.com.\\

\subsection{Тест 1}

\begin{tabular}{ccccccc}
\begin{cases}
$ 4x_1 - 3x_2 + x_3 - 5x_4 = 7$\\
$ x_1 - 2x_2 - 2x_3 - 3x_4 = 3$\\
$ 3x_1 - x_2 + 2x_3 = -1 $\\
$ 2x_1 + 3x_2 + 2x_3 - 8x_4 = -7 $\\
\end{cases}
\end{tabular}

Матрица является невырожденной и положительноопределённой.

\includegraphics[scale=0.8]{output.png}
Число обусловленности равно $7.5$. Определитель равен $135$. Решение системы методом Гаусса с выбором главного элемента полностью совпадает с решением, которое выдаёт wolframalfa. Метод верхней релаксации достигает точности $10^{-5}$ быстрее всего за 27 итераций при параметре $\omega = \frac{2}{3}$.
\newpage
Случай, когда матрица вырождена, программа также обрабатывает корректно:
\subsection{Тест 2}
\begin{tabular}{ccccccc}
\begin{cases}
$ 2x_1 - x_2 + 3x_3 +4x_4 = 5 $\\
$ 4x_1 - 2x_2 + 5x_3 + 6x_4 = 7 $\\
$ 6x_1 - 3x_2 + 7x_3 + 8x_4 = 9 $\\
$ 8x_1 - 4x_2 + 9x_3 + 10x_4 = 11 $\\
\end{cases}
\end{tabular}

\includegraphics{output2.png}
\newpage
\subsection{Тест 3}
\begin{tabular}{ccccccc}
\begin{cases}
$ x_1 + 2x_2 + 4x_3 = 31 $\\
$ 5x_1 + x_2 + 2x_3 = 29 $\\
$ 3x_1 - 1x_2 + x_3 = 10 $\\
\end{cases}
\end{tabular}

\includegraphics{output3.png}
\\
Число обусловленности равно $3.33333$. Определитель равен $-27$.
Решение методом Гаусса и определитель совпадают с решением wolframalpha.
\newpage
\subsection{Тест 4}
\begin{tabular}{ccccccc}
\begin{cases}
$ 24x_1 + 6x_2 + -12x_3 = 16 $\\
$ 6x_1 + 33x_2 + 6x_3 = 8 $\\
$ -12x_1 + 6x_2 + 24x_3 = 4 $\\
\end{cases}
\end{tabular}

\includegraphics{output4.png}\\
Число обусловленности равно $2.13$. Определитель равен $11664$.
Метод верхней релаксации достигает точности $10^{-5}$ быстрее всего за 29 итераций при параметре $\omega = \frac{4}{3}$.
\newpage
\subsection{Тест 5}
Матрица порядка $n = 50$
\begin{tabular}{ccccccc}
a_{ij} = 
\begin{cases}
$ q^{i+j}_{M} + 0.1 \cdot (j - i), i \neq j$\\
$ (q_{M} - 1)^{i + j}, i=j$\\
\end{cases}
\end{tabular}
$b_i = |x - \frac{n}{10}| \cdot i \cdot \sin{x}$\\
\includegraphics{output5.png}
\\
Столбец решение полностью не влез. Он совпадает с полученным с помощью wolframalpha решением.
Начальные параметры $q_M = 0.5, x \in (0, 1)$\\
Число обусловленности равно $5.781$. Определитель равен $5157.25$.
\newpage
\subsection{Тест 6}
\begin{tabular}{ccccccc}
\begin{cases}
$ 5x_1 + 3x_2 = 1 $\\
$ 2x_1 - x_2 = 0 $\\
\end{cases}
\end{tabular}

\includegraphics{output6.png}\\
Число обусловленности равно $2.(27)$. Определитель равен $-11$.
Метод верхней релаксации достигает точности $10^{-5}$ быстрее всего за 11 итераций при параметре $\omega = \frac{2}{3}$.
\newpage
\section{Выводы}

Метода Гаусса находит решение с высокой точностью, особенно, если использовать языки программирования с точной арифметикой.

Метод верхней релаксации даёт решение с высокой точностью уже всего через $10-20$ итераций, что будет даёт значительную выгоду по сравнению с методами Гаусса. Однако множество применимости этого метода сильно уже, чем у методов Гаусса.

\newpage


\end{document}